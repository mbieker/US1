\documentclass[11pt,ngerman,a4paper]{article}
%Gummi|061|=)
\usepackage{amsmath}
\usepackage{a4wide}
\usepackage{url}
\usepackage{amsthm}
\usepackage{amsbsy}
\usepackage{amssymb}
\usepackage{inputenc}
\usepackage{rotating} 
\usepackage{here}
\usepackage{graphicx}
\usepackage{paralist}
\usepackage{selinput}
\usepackage[separate-uncertainty=true]{siunitx}
\usepackage{booktabs}
\sisetup{}
\SelectInputMappings{%
adieresis={ä},
germandbls={ß},
}
\title{\textbf{Versuch US1: Grundlagen der Ultraschalltechnik}}
\author{Martin Bieker\\
		Julian Surmann\\
		\\
		Durchgef\"{u}hrt am 15.04.2014\\
		TU Dortmund}
\date{}
\usepackage{graphicx}
\begin{document}
\renewcommand\tablename{Tabelle}
\renewcommand\figurename{Abbildung}
\maketitle
\thispagestyle{empty}
\newpage
\clearpage
\setcounter{page}{1}


\section{Einleitung}
Ultraschall wird in vielen Bereichen der Medizin und Technik als bilgebendes Verfahren und zur zerst\"orugsfreifen Werkstoffuntersuchung eingesetzt. In diesem Versuch sollen die Grundlagen dieser Technik untersucht werden.
\section{Theorie}

Schallwellen im Frequenzbereich von \SI{20}{\hertz} bis \SI{1}{\giga\hertz} werden als Ultraschall bezeichnet. Schallwellen sind Druckschwankungen, welche sich als longitudinale Wellen durch ein Medium ausbreiten. Die Ausbreitungsgeschwindigkeit $c$ h\"angt dabei vom Medium ab, in dem sich die Welle befindet. Propagiert der Schall von einem Medium in ein Anderes treten f\"ur Wellen typische Effekte wie zum Beispiel Brechung und Reflexion auf. Dieses Verhalten an Grenzfl\"achen wird im Rahmen der Ultraschalltechnik zur Untersuchung eines K\"orpers verwendet. 
\subsection{Erzeugung und Detektion von Ultraschall}
In den meiten Anwendungen wird Ultraschall mit Hilfe des reziproken piezo-elektrischen Effektes erzeugt. Hierbei wird ein Kristall (zum Beispiel Quarz) durch ein angelegtes elektrisches Feld zu Schwingungen angeregt. Stimmt die Frequenz der Anregung mit der Eigenfrequenz des Kristalls \"uberein, so kommt es auf Grund von Resonanz zu einer Abstrahlung von Ultraschallwellen mit hoher Intensit\"at.  Die Umkehrung dieses Effektes wird auch zur Detektion des reflektierten Ultraschalls angewandt.
\subsection{Messverfahren} 
Die Messung kann entweder mit einer oder mit zwei Ultraschallsonden durchgef\"uhrt werden. Letzteres ist das so genannte Durchschallungssverfahren. Hierbei wird der von einem Empf\"anger erzeugte Ultraschall durch die Probe geschickt und von einem Empfa\"anger aufgezeichnet. Unregelm\"assigkeiten zeigen sich durch eine geringere Intensit\"at am Detektor. Um diese Fehlzstellen zu lokalisieren wird das Impuls-Echo-Verfahren verwendet. Hier wird eine Sonde sowohl als Empf\"anger und als Sender verwendet.
\subsection{Dastellung der Messergebnisse}
Es gibt mehrere M\"oglichkeiten die durch die Laufzeitmessung gewonnenen Daten darzustellen.
\paragraph{A-Scan} Beim so genannten Amplituden-Scan werden die gemessenen Amplituden in einen Diagramm gegen die Laufzeit aufgetragen. So k\"onnen eindimensionale Strukturen untersucht werden. 
\paragraph{B-Scan}
Beim Brightness-Scan wird ein zweidimensionales Schnittbild der untersuchten Struktur erzeugt. Dabei die Sonde gleichm\"assig \"uber das Objekt bewegt und die gemessenen Amplituden in verschiedenen Helligkeitsstufen dargestellt.  
\paragraph{TM-Scan} Beim Time-Motion-Scan werden laufend B-Scans in kurzen Abst\"anden erstellt, um Bewegungen innerhalb des untersuchten Objektes darzustellen

\section{Durchf\"uhrung }
\section{Auswertung}

\section{Quellen}
\begin{enumerate}[{[}1{]}]
\item Entnommen der Praktikumsanleitung \textit{} der TU Dortmund. Download am 21.04.14 unter:\\
 \url{http://129.217.224.2/HOMEPAGE/PHYSIKER/BACHELOR/AP/SKRIPT/V351.pdf}
\end{enumerate}

\section{Anhang}
\begin{itemize}
\item Tabellen
\item Auszug aus dem Messheft
\end{itemize}
\end{document}
