\documentclass[11pt,ngerman,a4paper]{article}
%Gummi|061|=)
\usepackage{amsmath}
\usepackage{a4wide}
\usepackage{url}
\usepackage{amsthm}
\usepackage{amsbsy}
\usepackage{amssymb}
\usepackage{inputenc}
\usepackage{rotating} 
\usepackage{here}
\usepackage{graphicx}
\usepackage{paralist}
\usepackage{selinput}
\usepackage[separate-uncertainty=true]{siunitx}
\usepackage{booktabs}
\sisetup{}
\SelectInputMappings{%
adieresis={ä},
germandbls={ß},
}
\title{\textbf{Versuch US1: Grundlagen der Ultraschalltechnik}}
\author{Martin Bieker\\
		Julian Surmann\\
		\\
		Durchgef\"{u}hrt am 22.04.2014\\
		TU Dortmund}
\date{}
\usepackage{graphicx}
\begin{document}
\renewcommand\tablename{Tabelle}
\renewcommand\figurename{Abbildung}
\maketitle
\thispagestyle{empty}
\newpage
\clearpage
\setcounter{page}{1}


\section{Einleitung}
Ultraschall wird in vielen Bereichen der Medizin und Technik als bildgebendes Verfahren und zur zerst\"orungsfreien Werkstoffuntersuchung eingesetzt. In diesem Versuch sollen die Grundlagen dieser Technik untersucht werden.
\section{Theorie}

Schallwellen im Frequenzbereich von \SI{20}{\hertz} bis \SI{1}{\giga\hertz} werden als Ultraschall bezeichnet. Schallwellen sind Druckschwankungen, welche sich als longitudinale Wellen durch ein Medium ausbreiten. Die Ausbreitungsgeschwindigkeit $c$ h\"angt dabei vom Medium ab, in dem sich die Welle befindet. Propagiert der Schall von einem Medium in ein Anderes treten f\"ur Wellen typische Effekte wie zum Beispiel Brechung und Reflexion auf. Dieses Verhalten an Grenzfl\"achen wird im Rahmen der Ultraschalltechnik zur Untersuchung eines K\"orpers verwendet. 
\subsection{Erzeugung und Detektion von Ultraschall}
In den meiten Anwendungen wird Ultraschall mit Hilfe des reziproken piezo-elektrischen Effektes erzeugt. Hierbei wird ein Kristall (zum Beispiel Quarz) durch ein angelegtes elektrisches Feld zu Schwingungen angeregt. Stimmt die Frequenz der Anregung mit der Eigenfrequenz des Kristalls \"uberein, so kommt es auf Grund von Resonanz zu einer Abstrahlung von Ultraschallwellen mit hoher Intensit\"at.  Die Umkehrung dieses Effektes wird auch zur Detektion des reflektierten Ultraschalls angewandt.
\subsection{Messverfahren} 
Die Messung kann entweder mit einer oder mit zwei Ultraschallsonden durchgef\"uhrt werden. Letzteres ist das so genannte Durchschallungssverfahren. Hierbei wird der von einem Empf\"anger erzeugte Ultraschall durch die Probe geschickt und von einem Empfa\"anger aufgezeichnet. Unregelm\"assigkeiten zeigen sich durch eine geringere Intensit\"at am Detektor. Um diese Fehlstellen zu lokalisieren wird das Impuls-Echo-Verfahren verwendet. Hier dient eine Sonde sowohl als Empf\"anger und als Sender. Diese sendet einen Ultraschall Impuls aus und misst danach die reflektierten Amplitude sowie deren Laufzeit. Diese Werte k\"onnen auf verschiedene Arten dargestellt werden.
\subsection{Dastellung der Messergebnisse}
Es gibt mehrere M\"oglichkeiten die durch die Laufzeitmessung gewonnenen Daten darzustellen.
\paragraph{A-Scan} Beim so genannten Amplituden-Scan werden die gemessenen Amplituden in einen Diagramm gegen die Laufzeit aufgetragen. So k\"onnen eindimensionale Strukturen untersucht werden. 
\paragraph{B-Scan}
Beim Brightness-Scan wird ein zweidimensionales Schnittbild der untersuchten Struktur erzeugt. Dabei die Sonde gleichm\"assig \"uber das Objekt bewegt und die gemessenen Amplituden in verschiedenen Helligkeitsstufen dargestellt.  
\paragraph{TM-Scan} Beim Time-Motion-Scan werden laufend B-Scans in kurzen Abst\"anden erstellt, um Bewegungen innerhalb des untersuchten Objektes darzustellen.

\section{Durchf\"uhrung }

\subsection{Messung der Schallgeschwindigkeit in Acryl}
Die Bestimmung der Schallgeschwindikeit durch eine Laufzeitmessung wird an drei zylinderf\"ormigen Proben durchgef\"uhrt. Es muss zuerst die Strecke, welche der Schall in den Proben zur\"ucklegt gemessen werden. Hierzu wird die H\"ohe der Zylinder mit Hilfe einer Schieblehre bestimmt.  
\subsubsection{Impuls-Echo-Verfahren}
Zur Messung der Laufzeit mittels Impuls-Echo-Verfahren wird eine \SI{2}{\mega\hertz}Ultraschall-Sonde an einen der Zylinder gekoppelt. Mit Hilfe eines A-Scans wird die Laufzeit in Abh\"angigkeit von der Zylinderh\"ohe bestimmt. Diese Messungen werden mit einer \SI{}{\mega\hertz} Sonde wiederholt. 
\subsubsection{Durchschallungsverfahren}
Bei dieser Messungen werden jeweils zwei \SI{2}{\mega\hertz} Sonden an die Strinseiten der Zylinder gekoppelt. Danach wird die Laufzeit des Schalls durch den Zylinder mit einem A-Scan ermittelt. Die Messung wird mit den anderen Zylindern wiederholt
\subsection{Untersuchung eines Akryblocks auf Fehlstellen}
\subsection{Vermessung eines  Augenmodells}
\section{Auswertung}
\subsection{Bestimmung der Schallgeschwindigkeit mit dem Impuls-Echo-Verfahren}
Die drei Acrylzylinder haben folgende Abmessungen:
 \begin{table}[h]
\centering
 \begin{tabular}{|c||c|c|c|}
 Zylinder & Zylinder 1 & Zylinder 2 & Zylinder 3 \\
  Länge [m] & 0.0397 & 0.0804 & 0.01205 \\
 \end{tabular}
\caption{Maße der Acrylzylinder}
\end{table}
\newline Mit Hilfe der Sonden wurden folgende Laufzeiten des Echos ermittelt:
 \begin{table}[h]
\centering
 \begin{tabular}{|c||c|c|c|}
 Zylinder & Zylinder 1 & Zylinder 2 & Zylinder 3 \\
 Laufzeit (2 MHz) [$10^{-6}s$] & 88.7 & 59.7 & 29.9 \\
 Laufzeit (1 MHz) [$10^{-6}s$] & 90  & 60.9 & 31.0 \\
 \end{tabular}
\caption{Laufzeitmessung mit Impuls-Echo-Verfahren}
\end{table}
\newline
Mit Hilfe der Formel ??? ergeben sich dann die Schallgeschwindigkeiten:
\begin{table}[h]
\centering
 \begin{tabular}{|c||c|c|c|}
 Zylinder & Zylinder 1 & Zylinder 2 & Zylinder 3 \\
 $v_{Schall}$ (2 MHz) [m/s] & 2656 & 2693 & 2717 \\
 $v_{Schall}$ (1 MHz) [m/s] & 2561  & 2640 & 2678 \\
 \end{tabular}
\caption{Schallgeschwindigkeit mit Impuls-Echo-Verfahren}
\end{table}
\newline
Allerdings weisen diese Schallgeschwindigkeiten einen systematischen Fehler auf. Mit Hilfe von linearer Regression wird die Funktion der Zeit in Abhängigkeit von der Strecke aufgetragen. Die Umkehrung der Steigung dieser Funktion ist die eigentliche Schallgeschwindigkeit. Der Laufzeitfehler ist mit dem Y-Achsenschnittpunkt der Funktion gegeben. Die Schallgeschwindigkeiten der beiden Messungen ergeben sich damit zu 2738 m/s und 2748 m/s.
Diese liegen mit einer Abweichung von weniger als 3 \%  an dem in der Literatur angegebenen Wert von 2670 m/s.
\subsection{Bestimmung der Schallgeschwindigkeit mit dem Durchschallungs-Verfahren}
Die Abmessungen der Acrylzylinder sind Abschnitt 4.1 zu entnehmen. Es ergeben sich folgende Werte für die Schallgeschwindigkeit:
\begin{table}[h]
\centering
 \begin{tabular}{|c||c|c|c|}
 Zylinder & Zylinder 1 & Zylinder 2 & Zylinder 3 \\
 $v_{Schall}$ (2 MHz) [m/s] & 2496 & 2619 & 2672 \\
 \end{tabular}
\caption{Schallgeschwindigkeit mit Impuls-Echo-Verfahren}
\end{table}
\subsection{Untersuchung eines Akrylblocks auf Fehlstellen}
\subsection{Biometrische Untersuchung eines Augenmodells}


\section{Quellen}
\begin{enumerate}[{[}1{]}]
\item Entnommen der Praktikumsanleitung \textit{} der TU Dortmund. Download am 21.04.14 unter:\\
 \url{http://129.217.224.2/HOMEPAGE/PHYSIKER/BACHELOR/AP/SKRIPT/V351.pdf}
\end{enumerate}

\section{Anhang}
\begin{itemize}
\item Tabellen
\item Auszug aus dem Messheft
\end{itemize}
\end{document}
