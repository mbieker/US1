\documentclass[11pt,ngerman,a4paper]{article}
%Gummi|061|=)
\usepackage{amsmath}
\usepackage{a4wide}
\usepackage{url}
\usepackage{amsthm}
\usepackage{amsbsy}
\usepackage{amssymb}
\usepackage{inputenc}
\usepackage{rotating} 
\usepackage{here}
\usepackage{graphicx}
\usepackage{paralist}
\usepackage{selinput}
\usepackage[separate-uncertainty=true]{siunitx}
\usepackage{booktabs}
\sisetup{}
\SelectInputMappings{%
adieresis={ä},
germandbls={ß},
}
\title{\textbf{Versuch US1: Grundlagen der Ultraschalltechnik}}
\author{Martin Bieker\\
		Julian Surmann\\
		\\
		Durchgef\"{u}hrt am 15.04.2014\\
		TU Dortmund}
\date{}
\usepackage{graphicx}
\begin{document}
\renewcommand\tablename{Tabelle}
\renewcommand\figurename{Abbildung}
\maketitle
\thispagestyle{empty}
\newpage
\clearpage
\setcounter{page}{1}


\section{Einleitung}
Ultraschall wird in vielen Bereichen der Medizin und Technik als bilgebendes Verfahren und zur zerst\"orugsfreifen Werkstoffuntersuchung eingesetzt. In diesem Versuch sollen die Grundlagen dieser Technik untersucht werden.
\section{Theorie}

Schallwellen im Frequenzbereich von \SI{20}{\hertz} bis \SI{1}{\giga\hertz} werden als Ultraschall bezeichnet. Schallwellen sind Druckschwankungen, welche sich als longitudinale Wellen durch ein Medium ausbreiten. Die Ausbreitungsgeschwindigkeit $c$ h\"angt dabei vom Medium ab, in dem sich die Welle befindet. Propagiert der Schall von einem Medium in ein Anderes treten f\"ur Wellen typische Effekte wie zum Beispiel Brechung und Reflexion auf. Dieses Verhalten an Grenzfl\"achen wird im Rahmen der Ultraschalltechnik zur Untersuchung eines K\"orpers verwendet. 
\subsection{Erzeugung und Detektion von Ultraschall}
In den meiten Anwendungen wird Ultraschall mit Hilfe des reziproken piezo-elektrischen Effektes erzeugt. Hierbei wird ein Kristall (zum Beispiel Quarz) durch ein angelegtes elektrisches Feld zu Schwingungen angeregt. Stimmt die Frequenz der Anregung mit der Eigenfrequenz des Kristalls \"uberein, so kommt es auf Grund von Resonanz zu einer Abstrahlung von Ultraschallwellen mit hoher Intensit\"at.  Die Umkehrung dieses Effektes wird auch zur Detektion des reflektierten Ultraschalls angewandt.
\subsection{Messverfahren} 
Die Messung kann entweder mit einer oder mit zwei Ultraschallsonden durchgef\"uhrt werden. Letzteres ist das so genannte Durchschallungssverfahren. Hierbei wird der von einem Empf\"anger erzeugte Ultraschall durch die Probe geschickt und von einem Empfa\"anger aufgezeichnet. Unregelm\"assigkeiten zeigen sich durch eine geringere Intensit\"at am Detektor. Um diese Fehlstellen zu lokalisieren wird das Impuls-Echo-Verfahren verwendet. Hier dient eine Sonde sowohl als Empf\"anger und als Sender. Diese sendet einen Ultraschall Impuls aus und misst danach die reflektierten Amplitude sowie deren Laufzeit. Diese Werte k\"onnen auf verschiedene Arten dargestellt werden.
\subsection{Dastellung der Messergebnisse}
Es gibt mehrere M\"oglichkeiten die durch die Laufzeitmessung gewonnenen Daten darzustellen.
\paragraph{A-Scan} Beim so genannten Amplituden-Scan werden die gemessenen Amplituden in einen Diagramm gegen die Laufzeit aufgetragen. So k\"onnen eindimensionale Strukturen untersucht werden. 
\paragraph{B-Scan}
Beim Brightness-Scan wird ein zweidimensionales Schnittbild der untersuchten Struktur erzeugt. Dabei die Sonde gleichm\"assig \"uber das Objekt bewegt und die gemessenen Amplituden in verschiedenen Helligkeitsstufen dargestellt.  
\paragraph{TM-Scan} Beim Time-Motion-Scan werden laufend B-Scans in kurzen Abst\"anden erstellt, um Bewegungen innerhalb des untersuchten Objektes darzustellen.

\section{Durchf\"uhrung }

\subsection{Messung der Schallgeschwindigkeit in Acryl}
Die Bestimmung der Schallgeschwindikeit durch eine Laufzeitmessung wird an drei zylinderf\"ormigen Proben durchgef\"uhrt. Es muss zuerst die Strecke, welche der Schall in den Proben zur\"ucklegt gemessen werden. Hierzu wird die H\"ohe der Zylinder mit Hilfe einer Schieblehre bestimmt.  
\subsubsection{Impuls-Echo-Verfahren}
Zur Messung der Laufzeit mittels Impuls-Echo-Verfahren wird eine \SI{2}{\mega\hertz}Ultraschall-Sonde an einen der Zylinder gekoppelt. Mit Hilfe eines A-Scans wird die Laufzeit in Abh\"angigkeit von der Zylinderh\"ohe bestimmt. Diese Messungen werden mit einer \SI{}{\mega\hertz} Sonde wiederholt. 
\subsubsection{Durchschallungsverfahren}
Bei dieser Messungen werden jeweils zwei \SI{2}{\mega\hertz} Sonden an die Strinseiten der Zylinder gekoppelt. Danach wird die Laufzeit des Schalls durch den Zylinder mit einem A-Scan ermittelt. Die Messung wird mit den anderen Zylindern wiederholt
\subsection{Untersuchung eines Akryblocks auf Fehlstellen}
Abbildung \ref{block} zeigt den zu untersuchenden Acrylblock. Zur Bestimmung von Lage und Gr\"o\ss e der Fehlstellen (Bohrungen) werden zun\"achst die Abmessungen des Blocks mit einer Schieblehre bestimmt. Danach wird eine \SI{1}{\mega\hertz} Sonde auf die Oberseite des Blocks gekoppelt und deren Abstand zu den einzelnen Fehlstellen durch einen A-Scan bestimmt. Dabei sind die im ersten Versuchsteil bestimmten Laufzeitfehler der Sonde zu ber\"usichtigen. Um die Gr\"\\ss e der Bohrungen zu bestimmen die Sonde an der entgegengesetzten Seite angekoppelt und der Abstand der Fehlstellen zur Unterseite durch einen weiteren A-Scan bestimmt. Da die Genauigkeit der Sonden von der Frequenz des Ultraschalls abh\"angen kann werden gerade durchgef\"uhrten Messungen mit einer \SI{}{\mega\hertz} Sonde wiederholt. 

\noindent
Danach wird um die Lage und Form der Bohrungen zu bestimmen ein B-Scan durchgef\"uhrt. Hierzu wird die Sonde gleichm\"assig und langsam \"uber den Acrylblock bewegt, damit am Messcomputer ein zweidimensionales Schnittbild erzeugt werden kann.

\subsection{Vermessung eines  Augenmodells}
Im letzten Versuchsteil sollen die Abmessungen eines Augenmodells bestimmt werden. Dazu wird die in Abbildung \ref{auge} gezeigt, eine \SI{2}{\mega\hertz} Sonde auf Hornhaut des Modells gekoppelt. Mit einem A-Scan werden die Reflexionen des Ultraschalls an Iris, Linse und Retina bestimmt.
\section{Auswertung}

\section{Quellen}
\begin{enumerate}[{[}1{]}]
\item Entnommen der Praktikumsanleitung \textit{} der TU Dortmund. Download am 21.04.14 unter:\\
 \url{http://129.217.224.2/HOMEPAGE/PHYSIKER/BACHELOR/AP/SKRIPT/V351.pdf}
\end{enumerate}

\section{Anhang}
\begin{itemize}
\item Tabellen
\item Auszug aus dem Messheft
\end{itemize}
\end{document}
